\documentclass[11pt,letterpaper,dvips]{article}
 
 
\usepackage{latexsym}
\usepackage{fancybox}
\usepackage{graphicx}
\usepackage{amssymb}
\usepackage{color}
%\usepackage{amstex}
\usepackage{ulem}
\usepackage{float}
 
\pretolerance=10000
\textwidth=7.0in
\textheight=9.6in
\voffset = -0.3in
\topmargin=0.0in
\headheight=0.00in
\hoffset = -0.3in
\headsep=0.00in
\oddsidemargin=0in
\evensidemargin=0in
\parindent=2em
\parskip=1.5ex
 
\def\rexp{{\rm e}}
\def\katzt{\ket{\alpha, t_0=0; t}}
\def\katt{\ket{\alpha, t_0; t}}
\def\usc{\mathcal{U}}
\def\tsc{\mathcal{T}}
\special{papersize=8.5in,11in}
\newcommand{\nid}{\noindent}
\newcommand{\ew}{W_\lambda}
\def\aap{A \& A}
\def\aj{AJ}
\def\apj{ApJ}
\def\apjl{ApJL}
\def\apjs{ApJS}
\def\apjsupp{ApJS}
\def\araa{ARAA}
\def\mnras{MNRAS}
\def\nat{Nature}
\def\pasp{PASP}
\def\hangpara{\par\hangindent 25pt\noindent}
\def\Lya{Ly$\alpha$}
\def\lya{Ly$\alpha$}
\def\Lyb{Ly$\beta$ }
\def\lyb{Ly$\beta$ }
\def\Lyg{Ly$\gamma$ }
\def\Lyd{Ly$\delta$ }
\def\etal{et al. }
\def\kms{km~s$^{-1}$ }
\def\mkms{{\rm \; km\,s\,^{-1}}}
\def\bmkms{{\rm\bf \; km\,s\,^{-1}}}
\def\L{$\lambda$}
\def\s-1{s$^{-1}$}
\def\Hz-1{Hz$^{-1}$}
\def\pohf{\rm p_\ohf}
\def\prhf{\rm p_\rhf}
\def\ohf{\frac{1}{2}}
\def\nohf{\frac{-1}{2}}
\def\rhf{\frac{3}{2}}
\def\sohf{\rm s_\ohf}
\def\esp{\Delta E_{2s, 2p}}
\def\chal{\frac{\Delta \alpha}{\alpha}}
\def\ylab{y_{\rm lab}}
\def\yaY{y \; {\rm and} \; Y}
\def\dadz{\frac{1}{\alpha} |\frac{{\rm d} \alpha}{{\rm d} z}|}
\def\abdadz{\frac{1}{\alpha} \left|\frac{{\rm d} \alpha}{{\rm d} z}\right|}
\def\abdadt{\frac{1}{\alpha} \left|\frac{{\rm d} \alpha}{{\rm d} t}\right|}
\def\sci#1{{\; \times \; 10^{#1}}}
\def\yold{y_{\rm old}}
\def\und#1{{\rm \underline{#1}}}
\def\dlmb{\Delta \lambda}
\def\abchal{| \chal |}
\def\lbar{\bar\lambda}
\def\elc{\rm e^-}
\def\qnn#1{\eqno {\rm {#1}}}
\def\l1l2{\lambda_2 \; {\rm and} \; \lambda_1}
\def\id{\indent}
\def\wid{\id\id}
\def\rid{\id\id\id}
\def\uid{\id\id\id\id}
\def\imp{\, \Rightarrow \>}
\def\di{\partial}
\def\ba{{\bf a}}
\def\bx{{\bf x}}
\def\bv{{\bf v}}
\def\bu{{\bf u}}
\def\bq{{\bf q}}
\def\br{{\bf r}}
\def\bs{{\bf s}}
\def\bp{{\bf p}}
\def\bj{{\bf j}}
\def\bk{{\bf k}}
\def\bm{{\bf m}}
\def\bn{{\bf n}}
\def\bd{{\bf d}}
\def\bg{{\bf g}}
\def\by{{\bf y}}
\def\bB{{\bf B}}
\def\bE{{\bf E}}
\def\bD{{\bf D}}
\def\bH{{\bf H}}
\def\bA{{\bf A}}
\def\bI{{\bf I}}
\def\bJ{{\bf J}}
\def\bK{{\bf K}}
\def\bL{{\bf L}}
\def\bS{{\bf S}}
\def\bF{{\bf F}}
\def\bN{{\bf N}}
\def\bP{{\bf P}}
\def\bR{{\bf R}}
\def\bU{{\bf U}}
\def\bM{{\bf M}}
\def\bY{{\bf Y}}
\def\bOm{{\bf \Omega}}
\def\bom{{\bf \omega}}
\def\bell{{\bf \ell}}
\def\di{\partial}
\def\ddi#1#2{\frac{\di #1}{\di #2}}
\def\gmm{\sqrt{1 - \frac{v^2}{c^2}}}
\def\eng{\varepsilon}
\def\dd#1#2{\frac{d #1}{d #2}}
\def\dt#1{\frac{d #1}{dt}}
\def\ddt{\frac{d}{d t}}
\def\ddit{\frac{\di}{\di t}}
\def\arr{\to \quad}
\def\smm{\sum\limits}
\def\intl{\int\limits}
\def\ointl{\oint\limits}
\def\liml{\lim\limits}
\def\prodl{\prod\limits}
\def\grd{{\bf \nabla}}
\def\crl{{\bf \grd} \times}
\def\dv{{\bf \grd} \cdot}
\def\ec{\frac{e}{c} \,}
\def\oc{\frac{1}{c}}
\def\xht{{\; \rm \hat x}}
\def\yht{{\; \rm \hat y}}
\def\zht{{\; \rm \hat z}}
\def\rht{{\; \rm \hat r}}
\def\pht{{\; \rm \hat \phi}}
\def\iht{{\; \rm \hat i}}
\def\jht{{\; \rm \hat j}}
\def\kht{{\; \rm \hat k}}
\def\theht{{\; \rm \hat \theta}}
\def\nht{{\; \rm \hat n}}
\def\oht{{\; \rm \hat 1}}
\def\tht{{\; \rm \hat 2}}
\def\vht{{\; \rm \hat v}}
\def\thht{{\; \rm \hat 3}}
\def\Rht{{\; \rm \hat R}}
\def\cyc{\Omega_c}
\def\dl{{d {\bf \ell}}}
\def\da{{\rm d {\bf a}}}
\def\ds{{\rm d {\bf s}}}
\def\smt{{\smll \rm T}}
\def\opi#1{\frac{1}{#1 \pi}}
\def\strs{\underline {\underline {\rm T}}}
\def\dtr{\, d^3 r}
\def\dftr{d^3 r \,}
\def\dtp{\, d^3 p}
\def\dftp{d^3 p \,}
\def\abrr{|\br - \br'|}
\def\rr{(\br - \br')}
\def\orr#1{\frac{#1}{\abrr}}
\def\trr#1{\frac{#1}{\abrr^3}}
\def\drr{\delta \rr}
\def\Grn{G(\br, \br')}
\def\Ylm{Y_{\ell m}}
\def\snth{\sin\theta}
\def\csth{\cos\theta}
\def\invr#1{\frac{1}{#1}}
\def\fpc{\frac{4 \pi}{c}}
\def\lgnl{P_\ell (\csth)}
\def\lgnn#1{{P_{#1} (\csth)}}
\def\thph{\theta, \phi}
\def\eht1{{\rm \hat e_1}}
\def\ehtb{{\rm \hat e_2}}
\def\iot{i \omega t}
\def\kr{\bk \cdot \br}
\def\ele{{\rm e^-}}
\def\iwc{\frac{i \omega}{c}}
\def\iwt{i \omega t}
\def\plr{{\rm \hat \epsilon}}
\def\skp{\smallskip}
\def\mkp{\medskip}
\def\bkp{\bigskip}
\def\elc{\rm e^{-s}}
\def\bmx{\left[\matrix}
\def\Det{{\rm Det \,}}
\def\bra#1{{<\negmedspace#1|}}
\def\ket#1{{|#1\negmedspace>}}
\def\kalp{\ket{\alpha}}
\def\kbet{\ket{\beta}}
\def\balp{\bra{\alpha}}
\def\bbet{\bra{\beta}}
\def\inpr#1#2
\def\otpr#1#2{{|#1><#2|}}
\def\dgg{\dagger}
\def\dual{\Leftrightarrow}
\def\hbm{\frac{\hbar^2}{2 m}}
\def\hbu{\frac{\hbar^2}{2 \mu}}
\def\mbh{\frac{2 m}{\hbar^2}}
\def\ih{\frac{i}{\hbar}}
\def\kjm{\ket{j m}}
\def\bjm{\bra{j m}}
\def\bjmp{\bra{j' m'}}
\def\ylm{Y_{\ell m}}
\def\pof#1{(\frac{#1}{2})}
\def\tqk{T_q^k}
\def\ptm{\frac{p^2}{2 m}}
\def\psn{\psi_n}
\def\tph{\frac{2 \pi}{\hbar}}
\def\plr{{\hat \epsilon}}
\def\Veff{V_{\rm eff}}
\def\rhs{\rho^{(s)}}
\def\rhis{\rho_i^{(s)}}
\def\tilH{\tilde H}
\def\tpi{2 \pi i}
\def\prin{\intl_P}
\def\ltk{\left [ \,}
\def\ltp{\left ( \,}
\def\ltb{\left \{ \,}
\def\rtk{\, \right  ] }
\def\rtp{\, \right  ) }
\def\rtb{\, \right \} }
\def\dc{\tilde d}
\def\bskp{\bigskip}
\def\nV{\Omega}
\def\d3x{d^3 x}
\def\mtI{\b{\b I}}
\def\cmma{\;\;\; ,}
\def\perd{\;\;\; .}
\def\delv{\Delta v}
\def\eikr{{\rm e}^{i \bk \cdot \br}}
\def\rmikr{{\rm e}^{i \vec k \cdot \vec r}}
\def\rmnikr{{\rm e}^{-i \vec k \cdot \vec r}}
\def\eikx{{\rm e}^{i \bk \cdot \bx}}
\def\muo{\mu_0}
\def\epo{\epsilon_0}
\def\rme{{\rm e}}
\def\dnu{\Delta \nu_D}
\def\rAA{\AA \enskip}
\def\halp{{\rm H_\alpha}}
\def\hbet{{\rm H_\beta}}
\def\N#1{{N({\rm #1})}}
\def\ekz{\rme^{i ( k_z z - \omega t )}}
\def\cm#1{\, {\rm cm^{#1}}}
\def\bcm#1{\; {\rm \mathbf{cm^{#1}}}}
\def\rmK{\; \rm K}
\def\rmm#1{\; {\rm #1}}
\def\Nperp{N_\perp}
\def\spkr#1{{\big \bf {#1}}}
\def\inA{\quad $\bullet$ \quad}
\def\inB{\quad\quad $\diamond$ \quad}
\def\inC{\quad\quad\quad $\circ$ \quad}
\def\sigfc{\sigma_{f_c}}
\def\sech{\rm sech}
\def\blck{\color[named]{Black}}
\def\blue{\color[named]{Blue}}
\def\skyblue{\color[named]{SkyBlue}}
\def\red{\color[named]{Red}}
\def\black{\color[named]{Black}}
\def\green{\color[named]{Green}}
\def\white{\color[named]{White}}
\def\yellow{\color[named]{Yellow}}
\def\purple{\color[named]{Purple}}
\def\maroon{\color[named]{Maroon}}
\def\redv{\color[named]{RedViolet}}
\def\mahog{\color[named]{Mahogany}}
\def\prb{$\frac{\delta \rho}{\rho}$}
\def\trd{\frac{2}{3}}
\def\mdn{\bar \rho}
\def\ss{v_s^2}
\def\msol{M_\odot}
\def\W#1{{W({\rm #1})}}
\def\Nav{N_a (v)}
\def\f#1{{f_{\rm #1}}}
\def\opt{$d \tau / d z \; $}
\def\btau{\bar\tau (v_{pk}) / \sigma (\bar \tau)}
\def\Ipk{I(v_{pk})/I_{c}}
\def\vrot{v_{rot}}
\def\sphr{\sqrt{R^2 + Z^2}}
\def\mbf{\mathbf}
\def\mgsqa{\, {\rm mag} / \Box ''}
\newcommand{\ialii}{Al{\sc II}}
\newcommand{\iari}{Ar{\sc I}}
\newcommand{\ini}{N{\sc I}}
\newcommand{\ioi}{O{\sc I}}
\newcommand{\iovi}{O{\sc VI}}
\newcommand{\iovii}{O{\sc VII}}
\newcommand{\ioviii}{O{\sc VIII}}
\newcommand{\ifeii}{Fe{\sc II}}
\newcommand{\ifeiii}{Fe{\sc III}}
\newcommand{\ispii}{S{\sc II}}
\newcommand{\iznii}{Zn{\sc II}}
\newcommand{\icaii}{Ca{\sc II}}
\newcommand{\inai}{Na{\sc I}}
\newcommand{\icii}{C{\sc II}}
\newcommand{\iciv}{C{\sc IV}}
\newcommand{\ihii}{H{\sc II}}
\newcommand{\ihi}{H{\sc I}}
\newcommand{\isiii}{Si{\sc II}}
\newcommand{\isiiv}{Si{\sc IV}}
\newcommand{\ici}{C{\sc I}}
\newcommand{\dcost}{d \negmedspace \cos\theta}
\newcommand{\mnvhi}{n_{\rm H {\sc I}}}
\newcommand{\nvhi}{$n_{\rm H {\sc I}}$}
\newcommand{\rem}{{\rm e}}
\def\otpr#1#2{{|#1><#2|}}
\def\avgq#1{<\negmedspace#1\negmedspace>}
\newcommand{\negm}{\negmedspace}
\newcommand{\dfhf}{^2\negm D_{\frac{5}{2}}}
\newcommand{\negt}{\negthinspace}
\newcommand{\nalph}{N/$\alpha$}
\newcommand{\alphh}{$\alpha$/H}
\newcommand{\nhi}{$N_{\rm HI}$}
\newcommand{\mnhi}{N_{\rm HI}}

\renewcommand{\descriptionlabel}[1]%
    {\hspace{\labelsep}\textsf{#1}}

\newenvironment{Aitemize}{%
 \renewcommand{\labelitemi}{$\bullet \;$}%
	\begin{itemize}}{\end{itemize}}

\newenvironment{Renumerate}{%
 \renewcommand{\theenumi}{\Roman{enumi}}
 \renewcommand{\labelenumi}{\theenumi)}
	\begin{enumerate}}{\end{enumerate}}

\newenvironment{Aenumerate}{%
 \renewcommand{\theenumi}{\Alph{enumi}}
 \renewcommand{\labelenumi}{\theenumi.}
	\begin{enumerate}}{\end{enumerate}}

\newenvironment{dmnd}{%
 \renewcommand{\labelitemi}{$\diamond \quad$}%
	\begin{itemize}}{\end{itemize}}

\newenvironment{Penumerate}{%
 \renewcommand{\labelenumi}{(\theenumi)}
	\begin{enumerate}}{\end{enumerate}}



\newcommand{\nnn}{ccd\#\#\#}
 
\special{papersize=8.5in,11in}

\renewcommand{\theenumi}{\Roman{enumi}}

\begin{document}
%\pagestyle{empty}
 

\begin{center}
{{\huge \bf Kast Data Reduction [v0.1]}}
\end{center}

\begin{enumerate}

{\Large   \item Kast Suggested Calibrations }
   \begin{itemize}
	\item 10 Bias (0s) frames
	\item 10 flat images (per setup and side)
	\item 2 Arc Calibrations (one per side)
	\item one Standard star \quad ({\it for each setup})
   \end{itemize}

{\Large   \item Pre-Reduction Setup }
   \begin{itemize}
	\item Computer Resources
	\begin{itemize}
	  \item $>$300 MHz processor
	  \item 1G disk space / night
	  \item 300M RAM
	  \item Linux or Solaris
	\end{itemize}
	\item Solaris WARNINGS (These do not apply to Linux)
	\begin{itemize}
	  \item Solaris+IDL does not release memory until IDL is exited
	  \item Consider exiting IDL occasionally and monitor memory usage
	  \item IDL+Solaris freezes up on the CPU intermittently
	  \begin{itemize}
		\item Possibly bug in licensing software
		\item Forced to kill IDL process and restart
	  \end{itemize}
	\end{itemize}
	\item Software
	\begin{itemize}
	  \item gcc or cc
	  \item IDL v5.4 or higher
	  \item IDL packages
		\begin{enumerate}
		  \item {\bf djs} IDL package  (Schlegel)
		  \item {\bf idlspec2d} IDL package  (Sloan public)
		  \item {\bf xidl} IDL package   (JXP)
		  \item {\bf Kast} Kast package [included with XIDL]  (JXP,GEP)
		\end{enumerate}
	\end{itemize}
	\item Log sheets [somewhat optional]
	\item This cookbook
   \end{itemize}

{\Large  \item Initial Setup (Repeat for each night) }

  \begin{enumerate}

    \item Create a new directory for the night (e.g.\ 06aug02) and enter it
    \item Create a 'Raw/' directory and put all the raw data in it.  
    \item If the filenames do not read b(r)\#\#\#\#.ccd, it is quite likely
	the code will be unhappy
    \item gzip the data (gzip *.fits)
    \item Launch idl in the directory above Raw/ (e.g. \ 06aug02)
    \item {\bf kast\_strct} :: Create the Kast structure.  
	\begin{itemize}
	\item This structure organizes
	the entire night of data and is the most important file created.  
	\item The routine creates a few things: (1) an IDL structure in memory with
	whatever name you choose (e.g.\ kast); 
	(2) the file 'kaststrct.fits' which is a fits
	version of the structure; (3) the file 'kast.list' which is an ASCII
	version of the fits file which lists the values of some of the tags.
	To view the fits structure outside of IDL, I 
	recommend the program 'fv' which I think stands for fitsview.  It allows
	you to examine binary fits tables.  
	\item If the NOEDIT keyword is not set, a
	gui will launch which allows some editing of the kast structure 
	(see below). \\
         \quad Example: IDL$> \;$ {\bf kast\_strct}, kast, /MKDIR, [/NOEDIT] \\  
         \quad Time   : $<$1s per image
	\end{itemize}


    \item {\bf kast\_editstrct} :: Modify the Kast structure.
	The previous step creates the structure and takes a guess at the 
	initial values of many of the tags based on the header card info.  It
	is difficult, however, to automate all of the values for the tags and
	therefore the user should carefully check the structure.  Also, the
	user should set flg\_anly = 0 for all of images which should
	be ignored during data reduction (bad flats, etc.).  For most of
	the important tags, one can use {\bf kast\_editstrct}.  The rest must
	be done from the command line by hand or through a simple
	IDL script (recommended). The obvious tags to modify are: 
	 \begin{itemize}
	  \item Obj :: Object name (this propagates into the final spectra 
		and {\bf should have no spaces!})
	  \item flg\_anly :: Include in analysis (defaulted to 1 for yes)
	  \item type :: OBJ, STD, DRK, TWI, QTZ, ARC, ZRO, ACQ, SLT, etc.  
	  \item setup :: 1L, 2L, etc.  (unique integer for each instrument
			setup including slit width)
	 \end{itemize}
         \quad Example: IDL$> \;$ {\bf kast\_editstrct}, kast \\  
         \quad Time   : User interaction

    \item Writing the Kast structure to disk:: In IDL you can modify the values
	of any of the tags.  You can then save the structure in fits form and 
	rewrite the ASCII file  with the routine {\bf kast\_wrstrct}. \\  
         \quad Example: IDL$> \;$ {\bf kast\_wrstrct}, kast, FITS='name' \\
         \quad Time   : fast

    \item Reading the Kast structure from disk:: {\bf kast\_ar}
	If no name is given, the file looks for the first fits 
	file starting 'kast' that contains a '\_'.  \\
         \quad Example: IDL$> \;$ kast = {\bf kast\_ar()} \\
         \quad Example: IDL$> \;$ kast = {\bf kast\_ar(}'kast\_name.fits'{\bf )} \\
         \quad Time   : fast

  \end{enumerate}

{\Large    \item Setup }
	\begin{itemize}
	  \item {\bf kast\_specsetup} ::
	This routine examines the kast structure and looks for calibration
	files associated with the various setups.
	It groups together exposures with identical Obj name and 
	sets the obj\_id tags accordingly.
	A summary of the Kast exposures is put in 'kast\_specsumm.txt'. \\
         \quad Example: IDL$> \;$ {\bf kast\_specsetup}, kast \\
         \quad Time   : Fast
	\end{itemize}


{\Large  \item Process Flats }
  \begin{itemize}
	\item This routines process the QTZ flats to create a 
	response image (pixel by pixel variations) for the CCD.

	\item {\bf kast\_mkflat} :: Creates a stacked, normalized QTZ flat 
	to correct pixel response variations.  
	The routine stacks by median averaging the images then normalizes
	by fitting a b-spline to the collapsed 1-D version of the stack. \\
         \quad Example: IDL$> \;$ {\bf kast\_mkflat}, kast, setup, [side]\\
         \quad Time   : 1s per flat 


  \end{itemize}

{\Large  \item Arc Images }
  \begin{itemize}
	\item {\bf kast\_mkarc} :: Process the Arcs.  
	This step combines Arc frames, flattens the resultant image and outputs
 	the final frame into the Arcs/ directory.  \\
         \quad Example: IDL$> \;$ {\bf kast\_mkarc}, kast, setup, [SIDE=side] \\
         \quad Time   : $<$5s per image


  \end{itemize}

\end{enumerate}

\clearpage

\begin{center}
{\Huge {\bf \quad 0. Extraction}}
\end{center}

  \begin{itemize}
	\item The following routines all apply to a single object, i.e.\
	multiple exposures of that object will be reduced together.
	\item Most of the following routines take the {\it kast} structure,
	and the {\it setup, side} and {\it obj\_id} tags. 
  \end{itemize}

\begin{enumerate}

{\Large  \item Process the Image}

  \begin{itemize}
	\item {\bf kast\_proc} :: Flat field the Raw image.
	This routine takes the index number of the kast structure as input
	or keywords setup and obj.
	This is the integer in the first column of the file 'kast.list'.
	Output is in 'Final/' and is a flattened flux and inverse variance fits file
	(one gzipped fits file with two exentsions per image). \\
         \quad Example: IDL$> \;$ {\bf kast\_proc}, kast, setup, side, obj\_id  \\
         \quad Time   : $<$5s per image \\
	 \quad Image Check: {\bf xatv}, 'Final/f\_b(r)\#.ccd'

  \end{itemize}

{\Large  \item Identify the Object}

  \begin{enumerate}
	\item {\bf kast\_fndobj} :: Allows the user to interactively identify
	the science objects (default is interactive selection).   It smashes a
	region of the spectrum and finds all peaks containing 4
	continuous pixels with 3 sigma significance.  The region is taken
	to be 20 pixels to either side of SCICLM, which can be set 
	explicity but by default is set to 700 and 515 for the 452/3306 and
	300/7500 grisings using the d46 splitter and 783 for the 452/3306 
	grising using the mirror.  For other setups there are no defaults and 
	SCICLM should be set explicitly.
	If not interactive, the program chooses the brightest object within
	5 pixels of the center of the slit.  Set the keyword /AUTO to 
	auto-identify the object and an aperture to mask for
	sky subtraction (not recommended).
	It then creates an object structure which
	stores the spectra of all objects identified in the slit.

\clearpage

	\begin{center}
	  {\Large {\bf specobjstrct} }
	\end{center}
	{\small
	\begin{tabular}{lcl}
	  \hline
	  Tag & Type & Comment \\
	  \hline
         field &  ' ' &  Name of field \\
         slit\_id &  0L & Used to store the order number (0-9) corresponding to
				physical order (15-6)\\  
         obj\_id &  ' ' & ID value (a=primary, b-z=serendip, x=NG) \\
         flg\_anly &  0 &  0=No analysis\\ 
         exp &  0. & \\
         xcen &  0L &  Column where obj was id\\
         ycen &  0. & \\
         flg\_aper &  0 &  0=boxcar\\
         aper &  fltarr(2) &  Widths of aperture, 0/1 = bottom/top (pixels)\\
         skyrms &  0. &  RMS of sky fit\\
         trace &  fltarr(5000) &\\
         npix &  0L & \\
         wave &  fltarr(5000) & \\
         fx &  fltarr(5000) & \\
         var &  fltarr(5000) &    $<=0$  rejected pix\\
         flg\_flux & 0 &    0=f$_\lambda$, 1=f$_\nu$\\
         flux &  fltarr(5000) &    Fluxed data\\
         sig &  fltarr(5000) &    Err in fluxed data\\
         date &  0.0d \\
         UT &  ' ' \\
         img\_fil &  ' ' \\
         slit\_fil &  ' ' \\
         instr\_strct &  ' ' & e.g. wfccdstr fits file\\
	  \hline
	\end{tabular}
	}

	\item The routine also does a crude trace to the object based on the
	identified object peak location.  This trace is used to mask out the
	object for sky subtraction.
	\item If the object to find is a standard star and is labeled STD in the 
	{\it kast} structure, the /STD flag should be used along with the 
	structure, setup and side.  This increases the default aperture size for 
	smashing in sky subtraction.
	\item Output is 'Extract/Obj\_{\it name}.fits' \\
         \quad Example: IDL$> \;$ {\bf kast\_fndobj}, kast, setup, side, obj\_id,
		[exp, REFWV= /NOCLOB, /AUTO, /CHK, /STD] \\
         \quad Time   : fast 

  \end{enumerate}


{\Large  \item Extraction}
  \begin{itemize}
	\item {\bf kast\_extract} ::
	This procedure performs an extraction of the object for each 
	exposure.  The defaut method uses the Sloan {\bf idlspec2d} package
	{\bf extract\_image}, which extracts by fitting a Gaussian profile to 
	the object along with a low order polynomial for the sky.  Setting 
	the keyword {\bf BOXCAR} causes the procedure to perform a 
	boxcar extraction of the object instead.  When using {\bf 
	BOXCAR} it's recommended to follow this procedure up with a 
	call to {\bf kast\_crays}, which identifies cosmic rays in the 
	spectra and alters their variance arrays accordingly (works only 
	when there exist multiple exposures of the same object).  \\ 

	\item  All of the output spectra are written to the Obj structure \\
	\item  A sky subracted image is added to the final image file 
	stored in the directory {bf Final}, use the keyword {\bf CHK} 
	to view the sky, sky-subtracted imgage and, in the default case,
	the y-model calculated by {\bf extract\_image}. \\
         \quad Example: IDL$> \;$ {\bf kast\_extract}, kast, setup, side, obj\_id, [/BOXCAR] \\
         \quad Example: IDL$> \;$ {\bf kast\_crays}, kast, setup, side, obj\_id \\
         \quad Time   : $<$10s per image
	\item Examine the final product: \\
         \quad IDL$> \;$ {\bf kast\_pltobj}, kast, setup, side, obj\_id, /noimg, /nowv \\
	 \quad {\bf OR} \\
	 \quad IDL$> \;$ {\bf xatv}, 'finalimage\_name', getsky$=$2 \\
  \end{itemize}

{\Large  \item Wavelength Calibrate Exposures (Required before fluxing images)} 
  \begin{itemize}
	\item {\bf kast\_wavesol} ::
	This procedure wavelength calibrates the extracted spectra of a given 
	object.  The default launches an interactive GUI to calibrate the 
	Arc image.  It is recommended to use the {\bf /auto} keyword, which 
	automates the process, is quite robust, and much less time consuming,  
	working by cross-correlating an archived wavelength solution with the 
	extracted ARC to 'reidentify' the lines.  It then performs a 
	fifth order fit to the lines. \\
         \quad Example: IDL$> \;$ {\bf kast\_wavesol}, kast, setup, side, obj\_id, /auto \\
         \quad Time   : $<$10s per image
         \quad Examine the output with:  
	    IDL$> \;$ {\bf kast\_pltobj}, kast, setup, side, obj\_id, /noimg \\
  \end{itemize}

{\Large  \item Combine Multiple Exposures }
  \begin{itemize}
	\item {\bf kast\_combspec} ::
	This procedure adds up multiple exposures of the same object,
	weighting by S/N after scaling each exposure by median ratio to the
	first.  It appropriately deals with rejected
	pixels (e.g.\ flagged CR's).  For objects
	with more than 2 exposures, it clips 5 sigma outliers as well.  Even
	if you have only a single exposure, you need to run this routine prior
	to fluxing and collapsing to a 1D spectrum as it creates the final
	spectrum structure. \\
	\item Output is a fits file in 'FSpec' named 'Fspec\_b(r)\#.ccd.gz'. \\
         \quad Example: IDL$> \;$ {\bf kast\_combspec}, kast, setup, side, obj\_id \\
         \quad Time   : fast \\
         \quad Examine the output with:  
	    IDL$> \;$ {\bf kast\_pltobj}, kast, setup, side, obj\_id, /noimg, /comb \\
  \end{itemize}

{\Large  \item Flux the Spectra}
  \begin{itemize}
	\item {\bf kast\_flux} ::
	This procedure fluxes each extracted spectra or combined spectra 
	for an object.  Generally,
	I'd suggest using the flux calibration that is defaulted. \\
	\item When the keyword {\bf /comb} is specified the program uses the 
	combined spectrum and overwrites the flux array in the file created by
	{\bf kast\_combspec}; this keyword also directs {\bf kast\_pltobj} 
	to use the combined spectrum for plotting. \\
         \quad Example: IDL$> \;$ {\bf kast\_flux}, kast, setup, side, obj\_id, [/comb] \\
         \quad Time   : fast \\
	\item Examine the final product: \\
         \quad IDL$> \;$ {\bf kast\_pltobj}, kast, setup, side, obj\_id, /flux,
		 [/noimg], [/comb] \\
  \end{itemize}

\end{enumerate}

\clearpage

\end{document}
	

